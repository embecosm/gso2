\documentclass{article}

\usepackage{todonotes}

\begin{document}

\title{GNU Superoptimizer 2: Design}
\author{James Pallister}

\maketitle

\section{Introduction}

\todo[inline]{mention original GSO here}

\todo[inline]{discuss superoptimization in general, and types of superoptimizers}
\todo[inline]{the need for a toolkit, which allows architectures and different types of superoptimziers to be constucted easily}

\section{Components}

\todo[inline]{overall discussion of how the components fit together}
\todo[inline]{What does the toolkit provide}

\subsection{Frontends}

\todo[inline]{Discussion of machine types}
\todo[inline]{what is needed from the instructions}
\todo[inline]{autogeneration of instructions}

\subsection{Slots}

\todo[inline]{concept of slots, different types of slots}
\todo[inline]{different ways of iterating over these slots}

\subsection{Testing}
\todo[inline]{ways of testing if a sequence is correct}

\section{Bruteforce superoptimizer example}

\todo[inline]{Give an example of constructing a superoptimizer}


\appendix
\section{Cost functions}
\todo[inline]{how to deal with different cost functions, and whether the search should end when a solution is found, etc}
\todo[inline]{for space, the searhc should end instantly}

\section{Peephole superoptimization}

\todo[inline]{changes required for peephole superoptimization}
\todo[inline]{harvesting the sequences}
\todo[inline]{changes to the testing routines}

\section{Stochastic superoptimization}
\section{Parallelisation}

\end{document}
